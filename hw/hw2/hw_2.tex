\documentclass[12pt]{article}
\usepackage{fancyhdr}
\usepackage{datetime}
\usepackage{enumitem}
\usepackage{float}
\usepackage{graphicx}
\graphicspath{ {images/} }
%\usepackage{showframe}

%custom variables
\newdate{date}{28}{10}{2016}
\newcommand{\hwNum}{2}
\newcommand{\assign}{Homework }
\newcommand{\class}{EC441: }
\newcommand{\name}{Daniel Andronov}


%header
\pagestyle{fancy}
\lhead{\name{}}
\chead{\thepage}
\rhead{\class{}\assign{}\hwNum{}}
\cfoot{\thepage}

\fancyheadoffset[LO,RE]{1pt}
\fancyheadoffset[RO,LE]{1pt}


%titlepage
\title{\class{}\assign{}\hwNum{}}
\author{\name{}}
\date{\displaydate{date}}


%addition settings
\topmargin=-0.45in
\evensidemargin=0in
\oddsidemargin=0in
\textwidth=6.5in
\textheight=9.0in
\headsep=0.25in

%adjusting vertical whitespace for the images at the top
\makeatletter
\setlength{\@fptop}{0pt}
\makeatother


%start the document
\begin{document}
\maketitle
\newpage

\section{Textbook Problems}
Problems from the book

\paragraph{Problem 2.23}
\begin{enumerate}[label=\textbf{Part \alph*)}]
	\item if $t_{dist} = \frac{NF}{u_s}$, then the server's upload time dominates the client's slowest download time.
	\item if $t_{dist} = \frac{F}{d_{min}}$, then ther client's slowest download time dominates the upload time of the server.
	\item The minium distriubution time cannot be faster that the larger of the server's upload time or the slowed client download time, and thus $t_dist \geq max\{ \frac{NF}{ u_s}, \frac{F}{ d_{min}} \}$.
\end{enumerate}

\paragraph{Problem 3.3}
\begin{tabular}{c*{9}{@{\,}c}}
	    & 0&1&0&1&0&0&1&1 \\
	    & 0&1&1&0&0&1&1&0 \\
	+ & 0&1&1&1&0&1&0&0 \\
	\hline
	& 0&0&1&0&1&1&0&1\\
	+ &&&&&&&&1\\
	\hline
	&0&0&1&0&1&1&1&0\\
\end{tabular}

\paragraph{Problem 3.13}
The only time that there are two packets on the link simultaneously is when there is an premature timeout on the side of the sender. In that situation, the sender will send the timed-out packet again, while the acknowledgement is still traveling to it. If packets can be re-ordered on the link, then the acknowledgement packet will arive before the re-sent packet. The receiver will discard the duplicate packet and the sender will accept the ack packet, and the outcome of the premature time out will be unaffected by the packet swap.

\paragraph{Problem 3.19}
INsert figure here

\paragraph{ Problem 3.22}
\begin{enumerate}[label=\textbf{Part \alph*)}]
	\item $k$ can be at either the front or the back of the window, e.g. $\{k-3, k-2, k-1, k\}$ or $\{k, k+1, k+2, k+3\}$, so the range should be $[k-3, k+3]$ or seven different numbers. 
	\item The maximum range of Ack values is equal to the sender's windwos size. This is because the receiver will only acknowledge the last received packet and does not acknowledge out of order packets. Thus, the receiver only needs to keep up with the packets in the sender's window.
\end{enumerate}

\paragraph{Problem 3.24}
\begin{enumerate}[label=\textbf{Part \alph*)}]
	\item 
	\item
	\item
	\item
\end{enumerate}
\section{Additional Questions}

\paragraph{Additional Question 1}
\begin{enumerate}[label=\textbf{Part \arabic*)}]
	\item 
		\begin{enumerate}[label=\textbf{\alph*)}]
			\item Stop \& Wait : 1 bit.
			\item GBN, W = 4: 3 bits
			\item SR, W = 4: 4 bits;
		\end{enumerate}
	\item For all of the following parts, let $T$ be the time between the sender sending a packet and receving its corresponding acknowledgement packet.
		\begin{enumerate}[label=\textbf{\alph*)}]
			\item $t_{total} = 8T - t_{prop}$. $T = t_{xmit} + 2t_{prop} = 1ms + 2 \times 1ms = 3ms$. So, $t_{total} = 23ms$.
			Insert diagram here
			\item $t_{total} = 4t_{xmit} + T + t_{xmit} + t_{prop} = 4 \times 1ms + 3ms + 1ms + 1ms = 9 m$. So, $t_{total} = 9ms$.
			\item Same as GBM so, 9ms.
		\end{enumerate}
	\item
		\begin{enumerate}[label=\textbf{\alph*)}]
			\item For stop \& wait, the time will be the same as before except the sender will wait 5ms for the time out, so $t_{total} = 28ms$.
			\item For GBN, after the fifth packet is sent, the timeout timer starts, after 5 ms, all the packets in the window, which will be packets 5,6,7,\& 8, will be send the receiver. So, $t_{total} = T + 5ms + 4t_{xmit} + t_{prop} = 13ms$.
			\item In the case of SR, because the receiver can buffer out of order packets and the packets that come acter 5 arrive before the fifth packet times out, the total time is simple the sum of T, the timeout length, and the packet delay. So, $t_{total} = 9ms$.
		\end{enumerate}
	\item
\end{enumerate}


\paragraph{Additonal Question 1}
Description of the first addition Question
\end{document}
This is never printed